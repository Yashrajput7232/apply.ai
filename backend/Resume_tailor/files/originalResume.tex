\documentclass[a4paper,10pt]{article}
\usepackage{geometry}
\geometry{margin=0.37in}
\usepackage{titlesec}
\usepackage{enumitem}
\usepackage{hyperref}

\setlist[itemize]{noitemsep, topsep=0pt}
\renewcommand{\baselinestretch}{1.0}
\titlespacing*{\section}{0pt}{4pt}{2pt}
\titlespacing*{\subsection}{0pt}{3pt}{1pt}

\begin{document}

\begin{center}
    {\LARGE \textbf{YASH DILIP RAJPUT}}\\
    \href{mailto:yashrajput02.scoe.comp@gmail.com}{yashrajput02.scoe.comp@gmail.com} \textbar
    \href{https://www.linkedin.com/in/yashrajput7232}{LinkedIn} \textbar
    \href{https://github.com/yash7232}{GitHub} \textbar
    +91-6281110280
\end{center}

\hrule
\vspace{3pt}
\section*{Summary}  
Software Engineer with expertise in \textbf{Java, Python, and Go}, specializing in \textbf{Spring Boot, Django, and microservices architecture}. Experienced in building \textbf{scalable, cloud-native applications} and optimizing production systems. Proficient in \textbf{DevOps, Kubernetes, Terraform, and CI/CD pipelines} to streamline deployments. Demonstrated ability to improve efficiency by solving \textbf{300+ coding problems} on platforms like LeetCode and CodeChef. Skilled in leveraging \textbf{AI and automation} to enhance workflows and optimize system performance.


\vspace{3 pt}
\hrule
\vspace{3pt}

\section*{Education}
\textbf{Bachelor of Engineering (B.E.) - Computer Engineering}\\
Pune University, Pune \hfill CGPA: 9.30/10
\vspace{3 pt}
\hrule
\vspace{3 pt}

\section*{Technical Skills}
\textbf{Programming}: Java, Python, C++ \textbar \space
\textbf{Backend}: Spring Boot, Flask, Django \textbar \space
\textbf{DevOps-cloud}:AWS (EKS, Lambda, S3), Terraform, Kubernetes, Docker \textbar \space
\textbf{Databases}: MongoDB, MySQL, PostgreSQL \textbar \space
\textbf{Tools-Others}: Git, Kafka, Prometheus, Grafana
\vspace{3}

\hrule
\vspace{3pt}

\section*{Work Experience}

\textbf{Bank of New York Mellon - Analyst} \hfill Jul 2024 -- Present\\
\textbf{Key Skills}: Java, Spring Boot, Python, Deployment, Backend, SQL, AI, LangChain, Git  
\begin{itemize}
    \item Worked on the \textbf{Complex Fund Reporting and Fund Admin Financial Reporting Tool}, a \textbf{Java Spring Boot microservices} platform, \textbf{reducing financial reporting errors by 50\%} for SEC, NECN, and NPORT compliance.  
    \item \textbf{Automated fund onboarding} with a \textbf{Generative AI-powered tool}, decreasing manual configuration time from \textbf{5 hours to 30 minutes per fund} (a \textbf{90\% reduction}).  
    \item Developed a \textbf{GenAI-based production issue analysis tool} that analyzes logs and suggests fixes, \textbf{accelerating incident resolution by 70\%} and reducing downtime.  
    \item \textbf{Worked on Optimization of real-time fund reporting} by integrating \textbf{Kafka} for event-driven architecture, \textbf{cutting report generation time by 60\%} and improving system scalability.  
    \item Led deployment enhancements using \textbf{Docker, Kubernetes, and CI/CD pipelines}, \textbf{reducing deployment failures by 90\%} and ensuring 99.9\% uptime.
    % \vspace{3 pt}
\end{itemize}
\vspace{3 pt}
\textbf{Bank of New York Mellon - Winter Intern} \hfill Jan 2024 -- Jun 2024\\
\textbf{Key Skills}: Java, Spring Boot, Spring MVC, REST API, Kafka, Redis, CI/CD, Docker, GitLab  
\begin{itemize}
    \item Designed \textbf{FAST (Fund Admin Support Tool)}, automating 40\% of manual tasks in \textbf{incident management}, reducing \textbf{average resolution time .}  
    \item Developed a \textbf{real-time fund tracking dashboard} with \textbf{Spring Boot and Angular}, improving \textbf{fund tracking accuracy by 98\%} and  integrated a \textbf{GEN-AI chat bot }to actively solve queries of the users which was trained on the Whole application.
    \item Enhanced \textbf{CI/CD pipelines with GitLab and Docker}, ensuring \textbf{zero-downtime deployments} and minimizing rollback occurrences.  
\end{itemize}
\vspace{3 pt}
\textbf{Shyena Tech Yarns - Software Engineer Intern} \hfill Oct 2023 -- Dec 2023\\
\textbf{Key Skills}: Python, Django, REST API, Celery, Stripe API, AI/ML  
\begin{itemize}
    \item Built an \textbf{AI-powered call analytics tool} that extracted real-time insights from customer conversations, \textbf{reducing customer support issue resolution time by 50\%}.  
    \item Developed \textbf{Django REST APIs}, enabling seamless \textbf{data retrieval with a 30\% improvement in response times}.  
    \item Optimized \textbf{Celery-based background task execution}, \textbf{increasing system efficiency by 40\%}.  
\end{itemize}
\vspace{3 pt}


\hrule
\section*{Projects}
\textbf{AI-Powered Job Search Assistant}  
\begin{itemize}
    \item Automated job search by integrating \textbf{real-time job scraping with Kafka}, \textbf{reducing manual effort by 70\%}.  
    \item Developed an \textbf{AI-driven job ranking algorithm}, improving \textbf{job-match accuracy by 80\%}.  
    \item \textbf{Generated personalized cover letters and recruiter emails}, boosting \textbf{response rates by 60\%}.  
    \item Created an \textbf{ATS-optimized resume generator}, Which aligns with the job description and the skill of candidate \textbf{increasing resume shortlisting probability by 50\%}.  
\end{itemize}

\vspace{3 pt}

\hrule
\vspace{3pt}

\section*{Achievements \& Certifications}
\begin{itemize}
    \item \textbf{Finalist - Kavach 2023 Hackathon}, contributed to cybersecurity solutions.
    \item \textbf{Published Research Paper in Explainable AI (XAI)} - \href{https://ijarsct.co.in/Paper18344.pdf}{IJARSCT}.
    \item \textbf{Contributed to Open Source} - HacktoberFest 2022 participant.
\end{itemize}

\hrule
\vspace{3pt}

\section*{Extracurricular Activities}
\begin{itemize}
    \item \textbf{Technical Blogging} on DevOps \& Cloud - \href{https://medium.com/@yash.7232.rajput}{Medium}.
\end{itemize}

\end{document}